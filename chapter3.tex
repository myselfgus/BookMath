%%%%%%%%%%%%%%%%%%%%% chapter3.tex %%%%%%%%%%%%%%%%%%%%%%%%%%%%%%%%%
% Capítulo 3: Representação Matemática da Linguagem e da Mente
%%%%%%%%%%%%%%%%%%%%%%%% Springer Nature %%%%%%%%%%%%%%%%%%%%%%%%%%

\chapter{Representação Matemática da Linguagem e da Mente}
\label{chap:representacao}

\section{Representação Vetorial de Frases e Sentenças}

A linguagem, como um sistema complexo e dinâmico, manifesta estados emocionais, cognitivos e sintáticos em cada frase ou sentença proferida. Ao representar essa linguagem em um espaço vetorial multidimensional, podemos analisar e quantificar esses padrões de forma precisa, abrindo portas para a compreensão profunda da relação entre linguagem e mente no contexto da psiquiatria.

\subsection{Definição do Espaço Vetorial Multidimensional}

Imaginemos um espaço vetorial $\mathbb{R}^n$, onde cada dimensão representa uma característica específica da linguagem, seja ela emocional, cognitiva ou sintática. Essas dimensões atuam como coordenadas que nos permitem posicionar cada frase ou sentença nesse espaço, criando um mapa multifacetado da linguagem. Algumas das dimensões que podemos considerar incluem:

\begin{itemize}
\item $v_1$: \textbf{Valência emocional}: Varia de altamente negativa (tristeza, raiva) a altamente positiva (alegria, entusiasmo), capturando o tom emocional geral da frase.
\item $v_2$: \textbf{Excitação emocional}: Reflete o nível de ativação emocional, variando de calmo e tranquilo a agitado e excitado.
\item $v_3$: \textbf{Dominância emocional}: Indica o grau de controle ou influência percebido na situação, variando de passivo e submisso a dominante e assertivo.
\item $v_4$: \textbf{Complexidade sintática}: Mede a sofisticação da estrutura gramatical da frase, considerando o número de orações subordinadas, uso de tempos verbais complexos e outras características sintáticas.
\item $v_5$: \textbf{Foco temático}: Associa a frase a um ou mais tópicos principais do discurso, permitindo identificar áreas de interesse ou preocupação do paciente.
\item $v_6$: \textbf{Intensidade afetiva}: Avalia a força da emoção expressa na frase, independentemente de sua valência (positiva ou negativa).
\item $v_7$: \textbf{Polaridade}: Classifica a frase como positiva, negativa ou neutra, oferecendo uma visão geral do sentimento expresso.
\item $v_8$: \textbf{Coerência narrativa}: Mede a fluidez e consistência da narrativa, avaliando se a frase se encaixa logicamente no contexto do discurso.
\item $v_9$: \textbf{Perspectiva temporal}: Analisa o uso de tempos verbais para identificar se a frase se refere ao passado, presente ou futuro, revelando a orientação temporal do paciente.
\item $v_{10}$: \textbf{Dissonância cognitiva}: Detecta discrepâncias entre o conteúdo verbal da frase e as emoções expressas, sinalizando possíveis conflitos internos ou tentativas de ocultar sentimentos.
\end{itemize}

Cada frase ou sentença $F_i$ pode ser mapeada para um vetor $\vec{F}_i$ nesse espaço multidimensional:

\begin{equation}
\vec{F}_i = (v_1, v_2, v_3, v_4, v_5, v_6, v_7, v_8, v_9, v_{10})
\end{equation}

Essa representação vetorial nos permite analisar múltiplas características da frase simultaneamente, criando um modelo rico e detalhado da linguagem que vai além da mera análise do conteúdo verbal.

\subsection{Produto Escalar: Medida de Similaridade}

O produto escalar entre dois vetores $\vec{F}_i$ e $\vec{F}_j$ nos oferece uma ferramenta poderosa para medir a similaridade entre duas frases. Quanto mais próximo de 1 for o valor do produto escalar normalizado, maior será a semelhança entre as frases em termos emocionais, cognitivos e sintáticos.

A fórmula do produto escalar é dada por:

\begin{equation}
\vec{F}_i \cdot \vec{F}_j = v_{1i} \cdot v_{1j} + v_{2i} \cdot v_{2j} + \cdots + v_{10i} \cdot v_{10j} = \sum_{k=1}^{10} v_{ki} \cdot v_{kj}
\end{equation}

E o produto escalar normalizado (cosseno da similaridade) é:

\begin{equation}
\cos(\theta) = \frac{\vec{F}_i \cdot \vec{F}_j}{||\vec{F}_i|| \cdot ||\vec{F}_j||} = \frac{\sum_{k=1}^{10} v_{ki} \cdot v_{kj}}{\sqrt{\sum_{k=1}^{10} v_{ki}^2} \cdot \sqrt{\sum_{k=1}^{10} v_{kj}^2}}
\end{equation}

\subsection{Distância Euclidiana: Medida de Divergência}

A distância euclidiana entre dois vetores $\vec{F}_i$ e $\vec{F}_j$ nos permite medir o grau de divergência entre duas frases. Frases com características emocionais, cognitivas ou sintáticas muito diferentes apresentarão uma maior distância euclidiana entre seus vetores.

A fórmula da distância euclidiana é:

\begin{equation}
d(\vec{F}_i, \vec{F}_j) = \sqrt{\sum_{k=1}^{10} (v_{ki} - v_{kj})^2}
\end{equation}

\section{Geometria das Emoções: Espaço Multidimensional}

As emoções humanas podem ser visualizadas como pontos ou regiões em um espaço multidimensional, onde cada dimensão representa um aspecto diferente da experiência emocional. Este modelo geométrico nos permite não apenas visualizar a relação entre diferentes estados emocionais, mas também analisar matematicamente suas propriedades e transformações.

\subsection{O Hipercubo Emocional}

Consideremos um espaço n-dimensional onde as emoções são representadas como pontos. Se utilizarmos as três dimensões básicas da emoção (valência, excitação e dominância), podemos visualizar um cubo 3D onde cada emoção ocupa uma posição específica:

\begin{itemize}
\item \textbf{Alegria}: alta valência, média-alta excitação, alta dominância $\rightarrow (0.8, 0.7, 0.9)$
\item \textbf{Tristeza}: baixa valência, baixa excitação, baixa dominância $\rightarrow (0.2, 0.3, 0.2)$
\item \textbf{Raiva}: baixa valência, alta excitação, alta dominância $\rightarrow (0.2, 0.9, 0.8)$
\item \textbf{Medo}: baixa valência, alta excitação, baixa dominância $\rightarrow (0.2, 0.8, 0.2)$
\end{itemize}

A distância no hipercubo emocional entre duas emoções $E_1$ e $E_2$ pode ser calculada como:

\begin{equation}
d(E_1, E_2) = \sqrt{\sum_{i=1}^{n} (e_{1i} - e_{2i})^2}
\end{equation}

\subsection{Variedades Emocionais como Subespaços}

As emoções não são pontos isolados, mas formam variedades contínuas (manifolds) no espaço emocional. Matematicamente, uma variedade emocional $\mathcal{M}$ pode ser representada como uma função parametrizada:

\begin{equation}
\mathcal{M}(t) = (f_1(t), f_2(t), \ldots, f_n(t))
\end{equation}

Por exemplo, a variedade da tristeza pode ser modelada como:

\begin{equation}
\mathcal{M}_{\text{tristeza}}(t) = (0.2 - 0.2t, 0.3 - 0.2t, 0.2, \ldots), \quad t \in [0,1]
\end{equation}

Onde valores maiores de $t$ correspondem a estados mais intensos de tristeza.

\section{Fluxo Euleriano na Dinâmica Mental}

O conceito de fluxo euleriano, originário da mecânica dos fluidos e teoria dos grafos, pode ser aplicado ao estudo da dinâmica mental para analisar como as emoções e cognições fluem e se transformam ao longo do tempo.

\subsection{Grafos de Transição Emocional}

Podemos representar os estados emocionais como nós em um grafo direcionado $G(V,E)$, onde cada aresta $e_{ij} \in E$ representa a possibilidade de transição do estado emocional $i$ para o estado $j$. O peso de cada aresta, $w(e_{ij})$, indica a probabilidade ou facilidade dessa transição.

Em termos matemáticos, para cada vértice $v \in V$:

\begin{equation}
\sum_{e_{iv} \in E} w(e_{iv}) = \sum_{e_{vj} \in E} w(e_{vj})
\end{equation}

Se esta condição for satisfeita, o grafo possui um fluxo euleriano, sugerindo uma dinâmica emocional bem equilibrada onde o paciente pode navegar entre diferentes estados emocionais de forma fluida e retornar ao equilíbrio.

\subsection{Equação de Continuidade para Estados Mentais}

Assim como na hidrodinâmica, podemos aplicar uma equação de continuidade para modelar o fluxo de estados mentais:

\begin{equation}
\frac{\partial \rho(\vec{x},t)}{\partial t} + \nabla \cdot \vec{J}(\vec{x},t) = S(\vec{x},t)
\end{equation}

Onde:
\begin{itemize}
\item $\rho(\vec{x},t)$ é a densidade de probabilidade de estar no estado mental $\vec{x}$ no tempo $t$
\item $\vec{J}(\vec{x},t)$ é o fluxo de probabilidade (corrente mental)
\item $S(\vec{x},t)$ representa fontes ou sumidouros (estímulos externos ou internos que geram ou absorvem estados mentais)
\end{itemize}

\section{Álgebra Linear da Linguagem}

A linguagem, como sistema simbólico, pode ser analisada através de diversas transformações lineares que revelam sua estrutura subjacente e suas propriedades emergentes.

\subsection{Decomposição em Valores Singulares (SVD) do Discurso}

O discurso de um paciente durante uma sessão terapêutica pode ser representado como uma matriz $M$ onde cada linha corresponde a uma frase e cada coluna a uma dimensão semântica ou emocional. Aplicando a decomposição em valores singulares:

\begin{equation}
M = U \Sigma V^T
\end{equation}

Onde:
\begin{itemize}
\item $U$ contém os vetores singulares à esquerda (padrões de frases)
\item $\Sigma$ é uma matriz diagonal contendo os valores singulares (importância de cada componente)
\item $V^T$ contém os vetores singulares à direita (dimensões semânticas/emocionais)
\end{itemize}

Esta decomposição revela a estrutura latente do discurso, permitindo identificar:
\begin{itemize}
\item Temas recorrentes (vetores dominantes em $U$)
\item Associações entre dimensões semânticas/emocionais (padrões em $V$)
\item A complexidade intrínseca do discurso (distribuição dos valores em $\Sigma$)
\end{itemize}
