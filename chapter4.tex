%%%%%%%%%%%%%%%%%%%%% chapter4.tex %%%%%%%%%%%%%%%%%%%%%%%%%%%%%%%%%
% Capítulo 4: Sistemas de Equações Dinâmicas Acopladas
%%%%%%%%%%%%%%%%%%%%%%%% Springer Nature %%%%%%%%%%%%%%%%%%%%%%%%%%

\chapter{Sistemas de Equações Dinâmicas Acopladas para Emoções e Cognições}
\label{chap:sistemas}

A complexidade da mente humana reside na interação dinâmica entre emoções e cognições. Em uma sessão psiquiátrica, esses estados não são estáticos, mas sim fluidos, influenciando-se mutuamente e respondendo a estímulos internos e externos. Para capturar essa dinâmica complexa, recorremos a sistemas de equações diferenciais acopladas, que nos permitem modelar a evolução temporal de emoções e cognições, revelando como esses estados se entrelaçam e reagem ao longo do tempo.

\section{Equações Diferenciais Acopladas para Emoções e Cognições}

Imagine o estado emocional de um paciente como um vetor $\vec{E}(t)$ que se transforma ao longo do tempo $t$. Da mesma forma, o estado cognitivo $\vec{C}(t)$ também é um vetor em constante mutação. A interação entre esses dois estados pode ser descrita por um sistema de equações diferenciais acopladas:

\begin{align}
\frac{d\vec{E}(t)}{dt} &= A_e \cdot \vec{E}(t) + B_e \cdot \vec{X}(t) + C_e \cdot \vec{C}(t) \\
\frac{d\vec{C}(t)}{dt} &= A_c \cdot \vec{C}(t) + B_c \cdot \vec{Y}(t) + C_c \cdot \vec{E}(t)
\end{align}

Onde:
\begin{itemize}
\item $\vec{E}(t)$: Vetor representando o estado emocional no tempo $t$.
\item $\vec{C}(t)$: Vetor representando o estado cognitivo no tempo $t$.
\item $A_e$ e $A_c$: Matrizes que descrevem a dinâmica interna dos estados emocional e cognitivo, respectivamente.
\item $B_e$ e $B_c$: Matrizes que representam a influência de eventos externos $\vec{X}(t)$ e $\vec{Y}(t)$ (interações verbais, eventos no ambiente).
\item $C_e$ e $C_c$: Matrizes de acoplamento que descrevem a influência mútua entre emoções e cognições.
\end{itemize}

\section{Interpretação das Equações}

Cada equação diferencial descreve como um estado (emocional ou cognitivo) se transforma ao longo do tempo, sob a influência de três fatores principais:

\subsection{Equação para Emoções}
\begin{itemize}
\item $A_e \cdot \vec{E}(t)$: O estado emocional anterior influencia sua própria evolução futura, representando a memória emocional e a tendência de persistir em um determinado estado.
\item $B_e \cdot \vec{X}(t)$: Eventos externos, como palavras ou acontecimentos, impactam diretamente o estado emocional, gerando alegria, tristeza, raiva ou outras emoções.
\item $C_e \cdot \vec{C}(t)$: O estado cognitivo influencia o estado emocional, mostrando como pensamentos e crenças moldam as emoções.
\end{itemize}

\subsection{Equação para Cognições}
\begin{itemize}
\item $A_c \cdot \vec{C}(t)$: O estado cognitivo anterior influencia sua própria evolução, refletindo a continuidade dos processos de pensamento e a tendência de manter crenças e ideias.
\item $B_c \cdot \vec{Y}(t)$: Estímulos cognitivos externos, como novas informações ou insights, podem alterar o estado cognitivo, levando a novas perspectivas ou reavaliações de crenças.
\item $C_c \cdot \vec{E}(t)$: O estado emocional influencia o estado cognitivo, mostrando como as emoções podem distorcer a percepção da realidade, influenciar a tomada de decisões e gerar pensamentos específicos.
\end{itemize}

\section{Acoplamento entre Emoções e Cognições}

O acoplamento entre emoções e cognições é o coração desse sistema de equações. Ele representa a intrincada dança entre esses dois estados, onde mudanças em um podem desencadear mudanças no outro, criando um ciclo de feedback complexo.

\begin{itemize}
\item \textbf{Matriz de Acoplamento $C_e$}: Captura como as cognições influenciam as emoções. Crenças negativas sobre si mesmo podem gerar tristeza, enquanto pensamentos otimistas podem levar à alegria.
\item \textbf{Matriz de Acoplamento $C_c$}: Representa como as emoções influenciam as cognições. A ansiedade pode levar a pensamentos catastróficos, enquanto a felicidade pode gerar pensamentos mais positivos e criativos.
\end{itemize}

Para analisar matematicamente este acoplamento, consideremos um sistema simplificado com uma emoção $e(t)$ e uma cognição $c(t)$:

\begin{align}
\frac{de}{dt} &= a_e \cdot e + c_e \cdot c \\
\frac{dc}{dt} &= a_c \cdot c + c_c \cdot e
\end{align}

As soluções deste sistema podem apresentar diferentes comportamentos dependendo dos parâmetros:

\begin{enumerate}
\item \textbf{Sistemas estáveis}: Se $a_e < 0$, $a_c < 0$ e $c_e \cdot c_c < a_e \cdot a_c$, o sistema eventualmente retorna ao equilíbrio após perturbações.

\item \textbf{Sistemas instáveis}: Se $a_e > 0$ ou $a_c > 0$ ou $c_e \cdot c_c > a_e \cdot a_c$, pequenas perturbações podem levar a grandes oscilações ou crescimento exponencial.

\item \textbf{Sistemas oscilatórios}: Se $a_e \cdot a_c < c_e \cdot c_c < 0$, o sistema pode apresentar oscilações periódicas (ciclos limite).
\end{enumerate}

\section{Análise de Estabilidade}

A estabilidade do sistema pode ser analisada examinando-se os autovalores da matriz do sistema. Para o sistema acoplado, precisamos considerar a matriz completa:

\begin{equation}
A = \begin{bmatrix} A_e & C_e \\ C_c & A_c \end{bmatrix}
\end{equation}

Os autovalores $\lambda_i$ de $A$ determinam o comportamento do sistema:
\begin{itemize}
\item Se todos os autovalores têm parte real negativa, o sistema é assintoticamente estável
\item Se algum autovalor tem parte real positiva, o sistema é instável
\item Se há autovalores com parte real zero, o sistema pode ser marginalmente estável ou instável
\end{itemize}

Em termos clínicos:
\begin{itemize}
\item \textbf{Autovalores negativos}: Indicam resiliência emocional e cognitiva, com tendência a retornar ao equilíbrio após perturbações
\item \textbf{Autovalores positivos}: Sugerem instabilidade, com tendência a amplificar pequenas perturbações, podendo levar a crises
\item \textbf{Autovalores complexos}: Indicam comportamento oscilatório, possivelmente refletindo ciclos de humor ou padrões ruminativos
\end{itemize}

\section{Aplicações Clínicas}

Esse modelo matemático oferece diversas aplicações clínicas na psiquiatria:

\begin{itemize}
\item \textbf{Modelagem de Crises Emocionais}: Identificamos condições que podem levar a uma escalada emocional, permitindo intervenções preventivas.
\item \textbf{Avaliação de Terapias}: Analisamos o impacto de intervenções cognitivo-comportamentais na matriz de acoplamento $C_e$, avaliando sua efetividade em reduzir a influência negativa das cognições sobre as emoções.
\item \textbf{Personalização de Terapias}: Identificamos padrões individuais de acoplamento entre emoções e cognições, permitindo a criação de intervenções terapêuticas personalizadas que visam quebrar ciclos de feedback negativo específicos para cada paciente.
\end{itemize}

\section{Extensões Não-Lineares}

O modelo linear apresentado é uma primeira aproximação, mas a dinâmica real da mente humana frequentemente exibe comportamentos não-lineares. Uma extensão natural é incluir termos não-lineares nas equações:

\begin{align}
\frac{d\vec{E}(t)}{dt} &= A_e \cdot \vec{E}(t) + B_e \cdot \vec{X}(t) + C_e \cdot \vec{C}(t) + F_e(\vec{E}(t), \vec{C}(t)) \\
\frac{d\vec{C}(t)}{dt} &= A_c \cdot \vec{C}(t) + B_c \cdot \vec{Y}(t) + C_c \cdot \vec{E}(t) + F_c(\vec{E}(t), \vec{C}(t))
\end{align}

Onde $F_e$ e $F_c$ são funções não-lineares que podem capturar fenômenos como:
\begin{itemize}
\item \textbf{Saturação}: quando uma emoção ou cognição atinge um limite máximo
\item \textbf{Efeitos de limiar}: mudanças súbitas de comportamento quando uma variável cruza um valor crítico
\item \textbf{Comportamentos caóticos}: extrema sensibilidade a condições iniciais, característica de certos estados mentais
\end{itemize}

Estas extensões não-lineares permitem modelar fenômenos mais complexos como transições de fase em estados mentais, bifurcações em trajetórias cognitivas, e emergência de estados alterados de consciência.
