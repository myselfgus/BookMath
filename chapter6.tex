%%%%%%%%%%%%%%%%%%%%% chapter6.tex %%%%%%%%%%%%%%%%%%%%%%%%%%%%%%%%%
% Capítulo 6: Análise Frequencial da Linguagem e das Emoções
%%%%%%%%%%%%%%%%%%%%%%%% Springer Nature %%%%%%%%%%%%%%%%%%%%%%%%%%

\chapter{Análise Frequencial da Linguagem e das Emoções}
\label{chap:fourier}

A linguagem, como um reflexo da mente humana, carrega em si uma sinfonia de emoções e cognições que se entrelaçam e evoluem ao longo do tempo. Para desvendar essa melodia complexa, recorremos à Transformada de Fourier, uma ferramenta matemática que nos permite decompor o sinal da linguagem em suas componentes de frequência, revelando os ritmos ocultos, as oscilações emocionais e cognitivas, e os padrões repetitivos que permeiam o discurso do paciente.

\section{A Transformada de Fourier: Conceito Básico}

A Transformada de Fourier atua como um prisma matemático, decompondo um sinal no tempo $L(t)$ (no nosso caso, a linguagem ou estados emocionais e cognitivos ao longo do tempo) em um espectro de frequências $F(\omega)$. Essa transformação nos permite analisar como as emoções ou cognições variam em diferentes escalas de tempo, desde mudanças rápidas e abruptas até oscilações lentas e graduais.

A fórmula da Transformada de Fourier é:

\begin{equation}
F(\omega) = \int_{-\infty}^{+\infty} L(t) e^{-i\omega t} dt
\end{equation}

E a Transformada Inversa:

\begin{equation}
L(t) = \frac{1}{2\pi} \int_{-\infty}^{+\infty} F(\omega) e^{i\omega t} d\omega
\end{equation}

Onde:
\begin{itemize}
\item $L(t)$: Sinal original no tempo, representando a linguagem ou um aspecto emocional/cognitivo.
\item $F(\omega)$: Transformada de Fourier, representando as componentes de frequência.
\item $\omega$: Frequência angular (radianos por segundo).
\item $e^{-i\omega t}$: Componente exponencial complexo que permite a decomposição em frequências.
\end{itemize}

\section{Aplicação na Análise de Emoções e Cognições}

Em uma sessão psiquiátrica, a linguagem do paciente pode ser vista como um sinal contínuo, onde emoções e cognições se manifestam em resposta a estímulos internos e externos. A Transformada de Fourier nos permite decompor esse sinal complexo, revelando os padrões subjacentes que moldam a experiência do paciente.

\subsection{Decomposição de Padrões Emocionais}

A linguagem emocional é como uma montanha-russa, com altos e baixos que refletem as flutuações do estado de espírito do paciente. A Transformada de Fourier nos permite decompor essa montanha-russa emocional em suas componentes de frequência, revelando:

\begin{itemize}
\item \textbf{Oscilações rápidas (alta frequência)}: Mudanças emocionais abruptas, como picos de ansiedade ou explosões de raiva.
\item \textbf{Oscilações lentas (baixa frequência)}: Mudanças emocionais graduais, como a transição da tristeza para a esperança ao longo da sessão.
\end{itemize}

Para uma série temporal de estados emocionais $E(t)$ medidos em $N$ pontos discretos $t_n$, podemos aplicar a Transformada Discreta de Fourier:

\begin{equation}
F_k = \sum_{n=0}^{N-1} E(t_n) e^{-i2\pi kn/N}
\end{equation}

O espectro de potência $P(f_k) = |F_k|^2$ indica a contribuição de cada frequência para o sinal emocional.

\subsection{Decomposição de Padrões Cognitivos}

Os processos cognitivos, como ruminação, foco temático e reavaliação, também podem ser analisados como sinais que variam ao longo do tempo. A Transformada de Fourier nos ajuda a identificar padrões de pensamento recorrentes e a entender como a mente do paciente se organiza e se transforma durante a sessão.

Se representarmos a atividade de uma dimensão cognitiva específica como uma função $C(t)$, sua Transformada de Fourier $\hat{C}(\omega)$ revelará se existe periodicidade nesta atividade:

\begin{equation}
\hat{C}(\omega) = \int_{-\infty}^{+\infty} C(t) e^{-i\omega t} dt
\end{equation}

Picos no espectro de potência $|\hat{C}(\omega)|^2$ em determinadas frequências $\omega_i$ indicariam ciclos de pensamento que se repetem em intervalos regulares $T_i = 2\pi/\omega_i$.

\section{Análise de Frequências Dominantes}

Após decompor o sinal da linguagem em suas componentes de frequência, podemos identificar quais frequências são mais proeminentes, revelando os padrões emocionais e cognitivos mais marcantes na experiência do paciente.

O espectro de frequência, um gráfico que mostra a magnitude de cada componente de frequência, nos permite visualizar quais frequências dominam o sinal. Picos no espectro indicam padrões repetitivos ou oscilações importantes.

Para identificar formalmente as frequências dominantes, podemos calcular os picos locais no espectro de potência:

\begin{equation}
\omega_{\text{dom}} = \{\omega_i : |\hat{L}(\omega_i)|^2 > |\hat{L}(\omega_{i-1})|^2 \text{ e } |\hat{L}(\omega_i)|^2 > |\hat{L}(\omega_{i+1})|^2 \text{ e } |\hat{L}(\omega_i)|^2 > \tau\}
\end{equation}

Onde $\tau$ é um limiar de significância.

\section{Transformada Wavelet: Análise Tempo-Frequência}

A Transformada de Fourier tem uma limitação: não preserva informação temporal. Para analisar como as frequências emocionais e cognitivas variam ao longo do tempo, recorremos à Transformada Wavelet.

A Transformada Wavelet Contínua (CWT) é definida como:

\begin{equation}
W(a,b) = \frac{1}{\sqrt{a}} \int_{-\infty}^{+\infty} L(t) \psi^*\left(\frac{t-b}{a}\right) dt
\end{equation}

Onde:
\begin{itemize}
\item $\psi$ é a wavelet mãe
\item $a > 0$ é o parâmetro de escala (inversamente relacionado à frequência)
\item $b$ é o parâmetro de translação (localização temporal)
\item $\psi^*$ é o complexo conjugado de $\psi$
\end{itemize}

A CWT produz um mapa bidimensional tempo-escala (ou tempo-frequência) que mostra como diferentes componentes de frequência evoluem ao longo do tempo.

Isso é particularmente útil para detectar:
\begin{itemize}
\item Momentos de transição emocional
\item Início e término de padrões ruminativos
\item Respostas emocionais transitórias a estímulos específicos
\item Mudanças no padrão cognitivo ao longo da sessão
\end{itemize}

\section{Assinatura Frequencial de Transtornos Mentais}

Diferentes transtornos mentais podem apresentar assinaturas características no domínio da frequência:

\begin{itemize}
\item \textbf{Transtorno Bipolar}: Oscilações de baixa frequência (ciclos longos) com alta amplitude no humor
\item \textbf{Ansiedade Generalizada}: Componentes de alta frequência predominantes, indicando variabilidade rápida no estado ansioso
\item \textbf{Depressão}: Espectro dominado por frequências muito baixas, refletindo estados emocionais persistentes e pouco reativos
\item \textbf{Transtorno de Personalidade Borderline}: Espectro amplo com componentes de alta e baixa frequência, indicando instabilidade em múltiplas escalas temporais
\end{itemize}

Essas assinaturas podem ser quantificadas pelo cálculo de métricas como:

\textbf{Índice de Dominância de Frequência}:
\begin{equation}
I_{DF} = \frac{\sum_{\omega < \omega_c} |F(\omega)|^2}{\sum_{\omega} |F(\omega)|^2}
\end{equation}

\textbf{Variância Espectral}:
\begin{equation}
\sigma^2_F = \frac{1}{N}\sum_{k=0}^{N-1} (\omega_k - \bar{\omega})^2 |F(\omega_k)|^2
\end{equation}

Onde $\bar{\omega} = \frac{\sum_{k=0}^{N-1} \omega_k |F(\omega_k)|^2}{\sum_{k=0}^{N-1} |F(\omega_k)|^2}$ é a frequência média ponderada.

\section{Aplicações Clínicas}

A Transformada de Fourier oferece um conjunto poderoso de ferramentas para a análise clínica em psiquiatria:

\begin{itemize}
\item \textbf{Identificação de Padrões Repetitivos}: Detectar ciclos emocionais em transtornos de humor ou padrões de ruminação cognitiva.
\item \textbf{Análise de Crises Emocionais}: Identificar sinais precoces de escalada emocional, permitindo intervenções preventivas.
\item \textbf{Avaliação de Intervenções}: Analisar se uma intervenção terapêutica resultou em mudanças nas frequências emocionais ou cognitivas, indicando sua efetividade.
\item \textbf{Personalização de Terapias}: Identificar padrões individuais de frequência, permitindo a criação de intervenções personalizadas que abordam as necessidades específicas de cada paciente.
\end{itemize}
