%%%%%%%%%%%%%%%%%%%%% chapter.tex %%%%%%%%%%%%%%%%%%%%%%%%%%%%%%%%%
% Fundamentos Matemáticos e Linguísticos da Psique
%%%%%%%%%%%%%%%%%%%%%%%% Springer Nature %%%%%%%%%%%%%%%%%%%%%%%%%%

\chapter{Fundamentos Matemáticos e Linguísticos da Psique}
\label{chap:fundamentos}

\abstract{Este capítulo propõe uma integração formal entre três domínios tradicionalmente segregados: a estrutura matemática do pensamento, a função constitutiva da linguagem na cognição, e os fundamentos neurofisiológicos da experiência subjetiva. Apresentamos um framework teórico-prático que formaliza a interrelação entre matemática e linguagem como fundamentos estruturantes da psique, fornecendo não apenas um modelo explanatório, mas também metodologias quantitativas para diagnóstico, intervenção e monitoramento psiquiátrico.}

\section{Introdução: A Interface Entre Linguagem, Matemática e Mente}
\label{sec:introducao}

A busca por um modelo unificado para compreensão da psique humana representa um dos mais complexos desafios epistemológicos da ciência contemporânea. Como observou Wittgenstein (1953/2009), ``os limites da minha linguagem significam os limites do meu mundo'' --- uma intuição filosófica que encontra crescente validação empírica nas neurociências modernas. Este ensaio propõe uma integração formal entre três domínios tradicionalmente segregados: a estrutura matemática do pensamento, a função constitutiva da linguagem na cognição, e os fundamentos neurofisiológicos da experiência subjetiva.

A fragmentação histórica entre abordagens qualitativas e quantitativas na psiquiatria criou o que Snow (1959/2012) denominou ``duas culturas'' epistemológicas --- uma divisão artificial que impede a formulação de modelos verdadeiramente integrais do funcionamento mental. Como argumenta Dehaene (2014), ``a matemática existe na mente antes da linguagem, mas encontra na linguagem sua expressão sistemática e desenvolvimento pleno'' --- sugerindo uma relação constitutiva que transcende a mera correlação.

Este trabalho apresenta um framework teórico-prático que formaliza a interrelação entre matemática e linguagem como fundamentos estruturantes da psique, fornecendo não apenas um modelo explanatório, mas também metodologias quantitativas para diagnóstico, intervenção e monitoramento psiquiátrico. Defendemos que esta integração representa o que Kuhn (1962/2020) caracterizaria como uma ``mudança de paradigma'' --- a emergência de um novo modelo que transcende e integra perspectivas anteriormente incompatíveis.

%------------------------------------------------------------------
\section{Fundamentos Ontológicos: A Psique Como Sistema Estruturado}
\label{sec:fundamentos-ontologicos}

\subsection{A Primazia da Linguagem na Estruturação da Experiência}
\label{subsec:primazia-linguagem}

A linguagem constitui não apenas um meio de comunicação, mas um sistema organizador da experiência humana. Como postula Sapir-Whorf (Whorf, 1956) em sua hipótese da relatividade linguística, ``a linguagem não é apenas um instrumento para reproduzir ideias, mas um modelador das mesmas'' --- perspectiva que encontra ressonância nas descobertas de Pulvermüller (2013) sobre como estruturas neurais específicas são recrutadas para processamento semântico e como estas influenciam a percepção.

A noção de que ``as coisas só existem quando nomeadas'' não representa mero construtivismo social, mas reflete o que Lakoff e Johnson (1980/2008) documentaram como o papel constitutivo das metáforas conceituais na cognição humana. No paradigma que propomos, esta função nomeadora é formalizada como um morfismo $T: \mathcal{M} \to \mathcal{L}$, que mapeia elementos do espaço mental $\mathcal{M}$ para o espaço linguístico $\mathcal{L}$, estabelecendo uma correspondência estrutural que preserva propriedades essenciais.

Evidências neurobiológicas corroboram esta perspectiva: estudos de neuroimagem funcional conduzidos por Fedorenko et al. (2011) demonstram sobreposição significativa entre redes neurais envolvidas no processamento linguístico e aquelas dedicadas a funções cognitivas de ordem superior, sugerindo que a linguagem não apenas expressa, mas estrutura o pensamento abstrato.

\subsection{A Estrutura Triádica da Realidade Mental}
\label{subsec:estrutura-triadica}

A psique humana apresenta uma organização que Peirce (1931/1974) caracterizaria como fundamentalmente triádica, composta por:

\begin{enumerate}
\item \textbf{Linguagem} como estrutura organizadora
\item \textbf{Cognição} como processamento operativo
\item \textbf{Emoção} como força motivadora e valorativa
\end{enumerate}

Esta tríade não representa mera taxonomia descritiva, mas um sistema dinâmico acoplado cujas interações podem ser formalizadas através das equações:

\begin{align}
\frac{dE(t)}{dt} &= A_e \cdot E(t) + B_e \cdot X(t) + C_e \cdot C(t) \label{eq:emocao}\\
\frac{dC(t)}{dt} &= A_c \cdot C(t) + B_c \cdot Y(t) + C_c \cdot E(t) \label{eq:cognicao}
\end{align}

Onde $E(t)$ representa o estado emocional, $C(t)$ o estado cognitivo, e os coeficientes matriciais $A_e$, $A_c$, $B_e$, $B_c$, $C_e$, $C_c$ modelam tanto dinâmicas internas quanto interações com fatores ambientais $X(t)$ e $Y(t)$.

Esta formalização matemática permite a análise do que Damasio (1994/2006) identifica como a indissociabilidade entre emoção e razão --- não como simples influência da emoção sobre a cognição, mas como um sistema único cuja dinâmica deve ser compreendida holisticamente.

%------------------------------------------------------------------
\section{Espaço Vetorial da Mente: Formalização Matemática}
\label{sec:espaco-vetorial}

\subsection{Axiomas do Espaço Mental}
\label{subsec:axiomas}

O estado mental pode ser representado como um vetor em um espaço de Hilbert $\mathcal{M}$, satisfazendo os seguintes axiomas:

\begin{enumerate}
\item \textbf{Completude}: Toda sequência de Cauchy $\{m_n\}$ em $\mathcal{M}$ converge para um elemento $m \in \mathcal{M}$
\item \textbf{Dimensionalidade Finita}: $\mathcal{M} = \text{span}\{\mathbf{e}_1(\text{afeto}), \mathbf{e}_2(\text{cognição}), \mathbf{e}_3(\text{volição}), \ldots, \mathbf{e}_n\}$
\item \textbf{Produto Interno}: $\langle F_1|F_2\rangle = \sum_i v_{i1} \cdot v_{i2}$, permitindo mensuração de similaridade
\item \textbf{Invariância Transformacional}: Certas propriedades permanecem invariantes sob transformações específicas
\end{enumerate}

Esta abordagem axiomática encontra precedentes no trabalho de von Neumann (1932/1955) sobre fundamentos matemáticos da mecânica quântica, e mais recentemente nas propostas de Aerts e Gabora (2005) para modelagem quântica de conceitos e estados cognitivos.

A representação vetorial permite a operacionalização do que James (1890/1983) descreveu como ``fluxo de consciência'' --- através da análise de trajetórias no espaço de fase mental, capturando o aspecto dinâmico e contínuo da experiência subjetiva.

\subsection{Dimensões Fundamentais do Espaço Mental}
\label{subsec:dimensoes}

O espaço mental compreende múltiplas dimensões ortogonais que, de acordo com análises fatoriais realizadas por Watson e Tellegen (1985) e expandidas por Russell (2003), podem ser agrupadas em três estruturas fundamentais:

\subsubsection{Estrutura Emocional}
\begin{enumerate}
\item \textbf{Valência Emocional} ($v_1$): Polarização positivo-negativa $[-1, 1]$
\item \textbf{Excitação Emocional} ($v_2$): Intensidade de ativação $[0, 1]$
\item \textbf{Dominância Emocional} ($v_3$): Percepção de controle $[0, 1]$
\item \textbf{Intensidade Afetiva} ($v_4$): Magnitude experiencial $[0, 1]$
\end{enumerate}

\subsubsection{Estrutura Cognitiva}
\begin{enumerate}
\item \textbf{Complexidade Sintática} ($v_5$): Elaboração estrutural do pensamento
\item \textbf{Coerência Narrativa} ($v_6$): Integração lógico-temporal
\item \textbf{Flexibilidade Cognitiva} ($v_7$): Adaptabilidade mental
\item \textbf{Dissonância Cognitiva} ($v_8$): Tensão entre elementos incompatíveis
\end{enumerate}

\subsubsection{Estrutura de Autonomia}
\begin{enumerate}
\item \textbf{Perspectiva Temporal} ($v_9$): Orientação passado-presente-futuro
\item \textbf{Autocontrole} ($v_{10}$): Capacidade inibitória e direcional
\end{enumerate}

Esta estruturação dimensional não apenas captura o que Scherer (2009) identifica como componentes essenciais da experiência emocional, mas também incorpora elementos de metacognição que Flavell (1979) demonstrou serem cruciais para o funcionamento executivo superior.

%------------------------------------------------------------------
\section{Análise Matemática de Estados Mentais}
\label{sec:analise-matematica}

\subsection{Sistema Dinâmico da Psique}
\label{subsec:sistema-dinamico}

O funcionamento mental pode ser modelado como um sistema dinâmico não-linear, seguindo princípios estabelecidos por Thelen e Smith (1994) em sua teoria dos sistemas dinâmicos do desenvolvimento cognitivo. Formalmente:

\textbf{Campo Vetorial Emocional}:
\begin{equation}
\mathbf{E}(x,y,z) = P(x,y,z)\mathbf{i} + Q(x,y,z)\mathbf{j} + R(x,y,z)\mathbf{k}
\label{eq:campo-vetorial}
\end{equation}

\textbf{Fluxo Emocional}:
\begin{equation}
\Phi = \iint_S \mathbf{E} \cdot d\mathbf{S}
\label{eq:fluxo}
\end{equation}

\textbf{Análise de Pontos Críticos}:
\begin{align}
\nabla \cdot \mathbf{E} &= 0 \quad \text{(conservativo)} \label{eq:conservativo}\\
\nabla \times \mathbf{E} &= 0 \quad \text{(irrotacional)} \label{eq:irrotacional}
\end{align}

Esta formalização permite identificar o que Zeeman (1976) denominou ``catástrofes'' --- transições abruptas no estado mental que caracterizam episódios psicopatológicos agudos. Tal abordagem encontra paralelo no trabalho de Prigogine (1984/1997) sobre estruturas dissipativas, oferecendo um modelo matemático para o que Jung (1928/1969) intuiu como processos de ``enantiodromia'' --- a tendência de sistemas psíquicos a produzirem seus opostos quando levados a extremos.

\subsection{Análise Espectral da Linguagem}
\label{subsec:analise-espectral}

A linguagem, como expressão do estado mental, pode ser analisada através de métodos espectrais que decompõem seus padrões em componentes fundamentais:

\textbf{Transformada de Fourier Mental}:
\begin{equation}
F(\omega) = \int_{-\infty}^{+\infty} L(t)e^{-i\omega t}dt
\label{eq:fourier}
\end{equation}

\textbf{Decomposição Wavelet}:
\begin{equation}
W(a,b) = \int L(t)\psi^*_{a,b}(t)dt
\label{eq:wavelet}
\end{equation}

Esta análise espectral permite identificar padrões linguísticos que, conforme demonstrado por Pennebaker et al. (2003), correlacionam-se significativamente com estados psicológicos específicos. A teoria matemática subjacente a estas transformadas foi desenvolvida por Mallat (1999) e encontra aplicações na identificação do que Gottschalk e Gleser (1969) denominaram ``marcadores linguísticos'' de estados psicopatológicos.

\subsection{Métricas e Medidas}
\label{subsec:metricas}

A quantificação de estados mentais e suas transformações requer métricas adequadas:

\textbf{Produto Escalar Mental}:
\begin{equation}
\langle F_1|F_2\rangle = \sum_i v_{i1} \cdot v_{i2}
\label{eq:produto-escalar}
\end{equation}

\textbf{Distância Psicológica} (baseada na métrica de Minkowski):
\begin{equation}
d(F_1,F_2) = \left(\sum_i|v_{i1} - v_{i2}|^p\right)^{1/p}
\label{eq:distancia}
\end{equation}
Para $p=2$, obtemos a distância Euclidiana.

\textbf{Divergência de Kullback-Leibler} (para distribuições probabilísticas de estados):
\begin{equation}
D_{KL}(P||Q) = \sum_i P(i) \log\frac{P(i)}{Q(i)}
\label{eq:kl-divergence}
\end{equation}

Estas métricas formalizam o que Shepard (1962/1987) demonstrou empiricamente: a percepção de similaridade psicológica obedece a princípios geométricos mensuráveis, permitindo a quantificação do que Kelly (1955/1991) denominou ``distância psicológica'' entre construtos pessoais.

%------------------------------------------------------------------
\section{Linguagem como Interface e Estrutura}
\label{sec:linguagem-interface}

\subsection{Função Organizadora da Linguagem}
\label{subsec:funcao-organizadora}

A linguagem desempenha papel fundamental na estruturação da experiência, não apenas como meio de comunicação, mas como o que Vygotsky (1934/1986) denominou ``ferramenta psicológica'' --- um instrumento de organização do pensamento. Esta função organizadora manifesta-se em três níveis:

\begin{enumerate}
\item \textbf{Nível Representacional}: Como sistema simbólico que, conforme Peirce (1931/1974), medeia a relação entre significante e significado
\item \textbf{Nível Estrutural}: Como gramática que, de acordo com Chomsky (1965/2014), impõe restrições formais ao pensamento
\item \textbf{Nível Narrativo}: Como estrutura temporal que, segundo Ricoeur (1984/1990), permite a integração coerente da experiência
\end{enumerate}

Evidências empíricas desta função organizadora são documentadas por Lupyan e Clark (2015), que demonstram como a categorização linguística modula a percepção até os níveis mais básicos de processamento visual.

\subsection{Operações Linguístico-Cognitivas}
\label{subsec:operacoes}

A interação entre linguagem e cognição pode ser formalizada através de operadores que transformam estados mentais:

\textbf{Operador Afetivo} ($\mathbf{A}$):
\begin{equation}
\mathbf{A} = \begin{pmatrix}
a_{11} & a_{12} & a_{13}\\
a_{21} & a_{22} & a_{23}\\
a_{31} & a_{32} & a_{33}
\end{pmatrix}
\label{eq:operador-afetivo}
\end{equation}

\textbf{Operador de Transformação Terapêutica} ($T$):
\begin{align}
T&: \mathcal{M}_1 \to \mathcal{M}_2 \label{eq:transformacao-terap}\\
||T(m)|| &\le k||m|| \nonumber
\end{align}

\textbf{Campo de Cura} ($\mathbf{V}$):
\begin{equation}
\mathbf{V}(x,y,z) = -\nabla\Phi(x,y,z)
\label{eq:campo-cura}
\end{equation}

Estes operadores formalizam o que Lakoff (1987) descreve como transformações cognitivas baseadas em metáforas conceituais, e o que White e Epston (1990) denominam ``reescritura narrativa'' em contextos terapêuticos.

\subsection{Estrutura Tensorial da Linguagem}
\label{subsec:estrutura-tensorial}

A linguagem, em sua complexidade multidimensional, pode ser representada através de estruturas tensoriais que capturam interdependências entre diferentes níveis linguísticos:

\begin{equation}
\text{Significado} = L_{ijk} e^i e^j e^k
\label{eq:tensor-linguistico}
\end{equation}

Onde $L_{ijk}$ representa o tensor linguístico e $e^i$, $e^j$, $e^k$ são vetores de base nos espaços léxico, sintático e pragmático.

Esta abordagem tensorial fundamenta-se no trabalho de Smolensky (1990) sobre arquiteturas neurais tensoriais e permite a modelagem do que Deacon (1997) descreve como a ``natureza hierárquica e multimodal da significação linguística''.

%------------------------------------------------------------------
\section{Metodologia Integrada Para Análise Psiquiátrica}
\label{sec:metodologia}

\subsection{Pipeline de Processamento Analítico}
\label{subsec:pipeline}

A implementação prática desta estrutura teórica requer um pipeline analítico que integra múltiplos níveis de processamento:

\begin{programcode}{Pipeline de Análise}
\begin{verbatim}
def analisar_sessao(discurso):
    # Extração de features
    vetores = extrair_vetores(discurso)

    # Análise espectral
    freq = fft(vetores)

    # Análise dimensional
    dims = analisar_dimensoes(vetores)

    # Integração
    return integrar_analises(freq, dims)
\end{verbatim}
\end{programcode}

Este pipeline implementa o que Marr (1982/2010) definiu como níveis complementares de análise: computacional (objetivo), algorítmico (método) e implementacional (realização).

\subsection{Biomarcadores Linguísticos}
\label{subsec:biomarcadores}

A identificação de estados mentais específicos pode ser operacionalizada através de biomarcadores linguísticos, conforme proposto por Tausczik e Pennebaker (2010). Estes incluem:

\begin{enumerate}
\item \textbf{Marcadores Léxicos}: Frequência de termos emocionais, cognitivos, perceptuais
\item \textbf{Marcadores Sintáticos}: Complexidade estrutural, organização temporal
\item \textbf{Marcadores Pragmáticos}: Funções comunicativas, posicionamento intersubjetivo
\end{enumerate}

A validade destes biomarcadores é corroborada por estudos longitudinais conduzidos por Stirman e Pennebaker (2001), que documentam alterações sistemáticas em padrões linguísticos precedendo crises psicopatológicas.

\subsection{Identificação de Estados Críticos}
\label{subsec:estados-criticos}

A detecção precoce de estados mentais potencialmente patológicos pode ser automatizada através da análise de campos gradientes:

\begin{programcode}{Detecção de Crises}
\begin{verbatim}
def detectar_crises(estado):
    # Cálculo de gradientes
    grad = calcular_gradiente(estado)

    # Análise de curvatura
    curv = calcular_curvatura(grad)

    # Detecção de pontos críticos
    return identificar_pontos_criticos(curv)
\end{verbatim}
\end{programcode}

Esta abordagem implementa o que Schiepek e Strunk (2010) denominam ``análise de transição de fase em sistemas psicológicos'', permitindo a identificação de pontos de bifurcação que frequentemente precedem descompensação psicopatológica.

%------------------------------------------------------------------
\section{Aplicações Clínicas: Da Teoria à Prática}
\label{sec:aplicacoes-clinicas}

\subsection{Diagnóstico Diferencial Quantitativo}
\label{subsec:diagnostico}

A aplicação de análise dimensional permite refinamento diagnóstico que transcende categorias nosológicas tradicionais. Como argumenta Insel et al. (2010) em sua defesa do Research Domain Criteria (RDoC), ``a heterogeneidade dentro das categorias diagnósticas e a comorbidade entre elas sugerem que dimensões subjacentes podem oferecer maior validade preditiva''.

Nossa abordagem oferece três níveis de avaliação diagnóstica:

\begin{enumerate}
\item \textbf{Mapeamento Dimensional Completo}: Quantificação vetorial em todas dimensões do espaço mental
\item \textbf{Análise de Trajetória}: Avaliação da evolução temporal do estado mental
\item \textbf{Identificação de Atratores Patológicos}: Detecção de padrões recorrentes disfuncionais
\end{enumerate}

Esta metodologia diagnóstica supera as limitações identificadas por Kendell e Jablensky (2003) quanto à validade de construto das categorias diagnósticas tradicionais, oferecendo o que Krueger e Markon (2006) denominam ``modelagem quantitativa hierárquica da psicopatologia''.

\subsection{Intervenções Terapêuticas Baseadas em Dados}
\label{subsec:intervencoes}

A formalização matemática do espaço mental permite modelagem preditiva de intervenções terapêuticas:

\textbf{Operador de Transformação Terapêutica} ($T$):
\begin{equation}
T: \mathcal{M}_1 \to \mathcal{M}_2
\end{equation}
Que mapeia estados patológicos para estados adaptativos.

\textbf{Campo de Cura} ($\mathbf{V}$):
\begin{equation}
\mathbf{V}(x,y,z) = -\nabla\Phi(x,y,z)
\end{equation}
Que direciona trajetórias mentais para atratores saudáveis.

Esta abordagem implementa o que Hayes et al. (2019) denominam ``processos de mudança baseados em evidências'', permitindo personalização de intervenções através do que Fisher (2015) caracteriza como ``modelagem de processo terapêutico em tempo real''.

\subsection{Monitoramento e Prevenção}
\label{subsec:monitoramento}

A implementação de sistemas de monitoramento contínuo permite intervenção preventiva baseada em sinais precoces:

\begin{enumerate}
\item \textbf{Tracking de Trajetórias Emocionais}: Monitoramento de evolução dimensional
\item \textbf{Detecção de Padrões Preditivos}: Identificação de precursores de crise
\item \textbf{Análise de Feedback Dinâmico}: Avaliação de resposta a intervenções
\end{enumerate}

Esta abordagem prevê o que Beaulieu-Prévost et al. (2017) documentam como ``janelas de oportunidade'' para intervenção precoce, potencialmente prevenindo o que Nelson et al. (2017) identificam como ``cascatas psicopatológicas'' --- processos em que alterações inicialmente sutis desencadeiam progressiva desorganização sistêmica.

%------------------------------------------------------------------
\section{Implicações Filosóficas e Científicas}
\label{sec:implicacoes}

\subsection{Epistemologia Matemática da Mente}
\label{subsec:epistemologia}

Esta integração entre matemática e psiquiatria levanta questões epistemológicas profundas. Como observa Lakatos (1976/2015), a matemática não é mera linguagem formal, mas um sistema de conhecimento com seu próprio desenvolvimento dialético. A aplicação de estruturas matemáticas à psique sugere o que Putnam (1975/1979) denominaria ``realismo matemático'' --- a noção de que certas estruturas formais não são meras convenções, mas refletem características fundamentais da realidade mental.

Tal perspectiva ressoa com o que Wigner (1960) caracterizou como ``a incompreensível eficácia da matemática nas ciências naturais'' --- um enigma filosófico que se torna ainda mais pronunciado quando estendido ao domínio da subjetividade humana.

\subsection{Ontologia da Linguagem e da Mente}
\label{subsec:ontologia}

A centralidade da linguagem na estruturação da psique sugere uma posição ontológica que transcende tanto o materialismo reducionista quanto o dualismo cartesiano. Como argumenta Taylor (1989) em suas ``Fontes do Self'', a linguagem constitui o meio através do qual a identidade humana se articula e se forma.

Nossa abordagem alinha-se com o que Merleau-Ponty (1945/2012) denominou ``corporeidade intencional'' --- reconhecendo tanto o substrato neurobiológico quanto a dimensão simbólica da experiência subjetiva como aspectos coessenciais da realidade mental.

\subsection{Implicações Éticas e Sociais}
\label{subsec:etica}

A capacidade de quantificar e modelar estados mentais levanta questões éticas significativas. Como adverte Lacan (1966/2006), o perigo da quantificação em psiquiatria é a redução do sujeito a objeto, e a substituição da compreensão pela classificação. Nossa abordagem reconhece o que Foucault (1963/1994) identificou como ``o olhar médico'' --- a tendência da medicina (inclusive psiquiátrica) a objetificar o paciente.

Para mitigar estes riscos, propomos integração do que Ricoeur (1990/1992) denomina ``ética hermenêutica'' --- abordagem que mantém a primazia da compreensão interpretativa mesmo quando utiliza ferramentas quantitativas. Esta perspectiva está alinhada com o que Jaspers (1913/1997) defendeu como a necessidade de complementaridade entre explicação causal e compreensão empática na psicopatologia.

%------------------------------------------------------------------
\section{Considerações Finais}
\label{sec:consideracoes-finais}

A síntese aqui proposta entre matemática, linguagem e psiquiatria oferece não apenas um framework teórico, mas uma metodologia operacionalizável para compreensão e intervenção clínica. Como observa Gould (1991/2015), ``a verdadeira novidade na ciência surge da integração criativa de perspectivas previamente separadas'' --- e esta integração representa precisamente tal síntese criativa.

A formalização matemática não substitui, mas complementa e potencializa a riqueza da compreensão fenomenológica da experiência humana. Como propõe Varela et al. (1991/2016) em ``A Mente Incorporada'', precisamos de uma ``neurofenomenologia'' --- abordagem que integra descrições em primeira pessoa da experiência subjetiva com análises em terceira pessoa de seus correlatos materiais.

O framework apresentado neste ensaio estabelece as bases para o que podemos denominar ``psiquiatria matemática'' --- não como reducionismo quantitativo, mas como expansão multidimensional de nossa capacidade de compreender, avaliar e transformar estados mentais patológicos em direção a maior integração, flexibilidade e bem-estar.

%------------------------------------------------------------------
\begin{thebibliography}{99}

\bibitem{aerts2005} Aerts, D., \& Gabora, L. (2005). A theory of concepts and their combinations I: The structure of the sets of contexts and properties. \textit{Kybernetes, 34}(1/2), 167--191.

\bibitem{beaulieu2017} Beaulieu-Prévost, D., Charnonneau, S., Jourdan A., et al. (2017). Predictive patterns of early symptom change in dialectical behavior therapy. \textit{Personality Disorders: Theory, Research, and Treatment, 8}(4), 317--327.

\bibitem{chomsky2014} Chomsky, N. (2014). \textit{Aspectos da teoria da sintaxe}. (Orig. 1965). Editora da Universidade de São Paulo.

\bibitem{damasio2006} Damasio, A. (2006). \textit{O erro de Descartes: Emoção, razão e o cérebro humano}. (Orig. 1994). Companhia das Letras.

\bibitem{deacon1997} Deacon, T. W. (1997). \textit{The symbolic species: The co-evolution of language and the brain}. W.W. Norton.

\bibitem{dehaene2014} Dehaene, S. (2014). \textit{Consciousness and the brain: Deciphering how the brain codes our thoughts}. Viking Press.

\bibitem{fedorenko2011} Fedorenko, E., Behr, M. K., \& Kanwisher, N. (2011). Functional specificity for high-level linguistic processing in the human brain. \textit{Proceedings of the National Academy of Sciences, 108}(39), 16428--16433.

\bibitem{fisher2015} Fisher, A. J. (2015). Toward a dynamic model of psychological assessment: Implications for personalized care. \textit{Journal of Consulting and Clinical Psychology, 83}(4), 825--836.

\bibitem{flavell1979} Flavell, J. H. (1979). Metacognition and cognitive monitoring: A new area of cognitive--developmental inquiry. \textit{American Psychologist, 34}(10), 906--911.

\bibitem{foucault1994} Foucault, M. (1994). \textit{O nascimento da clínica}. (Orig. 1963). Forense Universitária.

\bibitem{gottschalk1969} Gottschalk, L. A., \& Gleser, G. C. (1969). \textit{The measurement of psychological states through the content analysis of verbal behavior}. University of California Press.

\bibitem{gould2015} Gould, S. J. (2015). \textit{A falsa medida do homem}. (Orig. 1991). Martins Fontes.

\bibitem{hayes2019} Hayes, S. C., Hofmann, S. G., \& Ciarrochi, J. (2019). A process-based approach to psychological diagnosis and treatment. \textit{Clinical Psychology Review, 74}, 101778.

\bibitem{insel2010} Insel, T., Cuthbert, B., Garvey, M., et al. (2010). Research domain criteria (RDoC): Toward a new classification framework for research on mental disorders. \textit{American Journal of Psychiatry, 167}(7), 748--751.

\bibitem{james1983} James, W. (1983). \textit{The principles of psychology}. (Orig. 1890). Harvard University Press.

\bibitem{jaspers1997} Jaspers, K. (1997). \textit{General psychopathology}. (Orig. 1913). Johns Hopkins University Press.

\bibitem{jung1969} Jung, C. G. (1969). \textit{The structure and dynamics of the psyche}. (Orig. 1928). Princeton University Press.

\bibitem{kelly1991} Kelly, G. A. (1991). \textit{The psychology of personal constructs}. (Orig. 1955). Routledge.

\bibitem{kendell2003} Kendell, R., \& Jablensky, A. (2003). Distinguishing between the validity and utility of psychiatric diagnoses. \textit{American Journal of Psychiatry, 160}(1), 4--12.

\bibitem{krueger2006} Krueger, R. F., \& Markon, K. E. (2006). Reinterpreting comorbidity: A model-based approach to understanding and classifying psychopathology. \textit{Annual Review of Clinical Psychology, 2}, 111--133.

\bibitem{kuhn2020} Kuhn, T. S. (2020). \textit{A estrutura das revoluções científicas}. (Orig. 1962). Perspectiva.

\bibitem{lacan2006} Lacan, J. (2006). \textit{Écrits: The first complete edition in English}. (Orig. 1966). W. W. Norton.

\bibitem{lakatos2015} Lakatos, I. (2015). \textit{Proofs and refutations: The logic of mathematical discovery}. (Orig. 1976). Cambridge University Press.

\bibitem{lakoff1987} Lakoff, G. (1987). \textit{Women, fire, and dangerous things: What categories reveal about the mind}. University of Chicago Press.

\bibitem{lakoff2008} Lakoff, G., \& Johnson, M. (2008). \textit{Metaphors we live by}. (Orig. 1980). University of Chicago Press.

\bibitem{lupyan2015} Lupyan, G., \& Clark, A. (2015). Words and the world: Predictive coding and the language-perception-cognition interface. \textit{Current Directions in Psychological Science, 24}(4), 279--284.

\bibitem{mallat1999} Mallat, S. (1999). \textit{A wavelet tour of signal processing}. Academic Press.

\bibitem{marr2010} Marr, D. (2010). \textit{Vision: A computational investigation into the human representation and processing of visual information}. (Orig. 1982). MIT Press.

\bibitem{merleau2012} Merleau-Ponty, M. (2012). \textit{Fenomenologia da percepção}. (Orig. 1945). Martins Fontes.

\bibitem{nelson2017} Nelson, B., McGorry, P. D., Wichers, M., et al. (2017). Moving from static to dynamic models of the onset of mental disorder: A review. \textit{JAMA Psychiatry, 74}(5), 528--534.

\bibitem{peirce1974} Peirce, C. S. (1974). \textit{Collected papers of Charles Sanders Peirce}. (Orig. 1931). Harvard University Press.

\bibitem{pennebaker2003} Pennebaker, J. W., Mehl, M. R., \& Niederhoffer, K. G. (2003). Psychological aspects of natural language use: Our words, our selves. \textit{Annual Review of Psychology, 54}(1), 547--577.

\bibitem{prigogine1997} Prigogine, I. (1997). \textit{The end of certainty: Time, chaos, and the new laws of nature}. (Orig. 1984). Free Press.

\bibitem{pulvermuller2013} Pulvermüller, F. (2013). How neurons make meaning: Brain mechanisms for embodied and abstract-symbolic semantics. \textit{Trends in Cognitive Sciences, 17}(9), 458--470.

\bibitem{putnam1979} Putnam, H. (1979). \textit{Mathematics, matter and method: Philosophical papers}. (Orig. 1975). Cambridge University Press.

\bibitem{ricoeur1990} Ricoeur, P. (1990). \textit{Time and narrative}. (Orig. 1984). University of Chicago Press.

\bibitem{ricoeur1992} Ricoeur, P. (1992). \textit{Oneself as another}. (Orig. 1990). University of Chicago Press.

\bibitem{russell2003} Russell, J. A. (2003). Core affect and the psychological construction of emotion. \textit{Psychological Review, 110}(1), 145--172.

\bibitem{schiepek2010} Schiepek, G., \& Strunk, G. (2010). The identification of critical fluctuations and phase transitions in short term and coarse-grained time series. \textit{Biological Cybernetics, 102}(3), 197--207.

\bibitem{scherer2009} Scherer, K. R. (2009). The dynamic architecture of emotion: Evidence for the component process model. \textit{Cognition and Emotion, 23}(7), 1307--1351.

\bibitem{shepard1987} Shepard, R. N. (1987). Toward a universal law of generalization for psychological science. \textit{Science, 237}(4820), 1317--1323.

\bibitem{smolensky1990} Smolensky, P. (1990). Tensor product variable binding and the representation of symbolic structures in connectionist systems. \textit{Artificial Intelligence, 46}(1-2), 159--216.

\bibitem{snow2012} Snow, C. P. (2012). \textit{The two cultures}. (Orig. 1959). Cambridge University Press.

\bibitem{stirman2001} Stirman, S. W., \& Pennebaker, J. W. (2001). Word use in the poetry of suicidal and nonsuicidal poets. \textit{Psychosomatic Medicine, 63}(4), 517--522.

\bibitem{tausczik2010} Tausczik, Y. R., \& Pennebaker, J. W. (2010). The psychological meaning of words: LIWC and computerized text analysis methods. \textit{Journal of Language and Social Psychology, 29}(1), 24--54.

\bibitem{taylor1989} Taylor, C. (1989). \textit{Sources of the self: The making of the modern identity}. Harvard University Press.

\bibitem{thelen1994} Thelen, E., \& Smith, L. B. (1994). \textit{A dynamic systems approach to the development of cognition and action}. MIT Press.

\bibitem{varela2016} Varela, F. J., Thompson, E., \& Rosch, E. (2016). \textit{The embodied mind: Cognitive science and human experience}. (Orig. 1991). MIT Press.

\bibitem{vonneumann1955} von Neumann, J. (1955). \textit{Mathematical foundations of quantum mechanics}. (Orig. 1932). Princeton University Press.

\bibitem{vygotsky1986} Vygotsky, L. S. (1986). \textit{Thought and language}. (Orig. 1934). MIT Press.

\bibitem{watson1985} Watson, D., \& Tellegen, A. (1985). Toward a consensual structure of mood. \textit{Psychological Bulletin, 98}(2), 219--235.

\bibitem{white1990} White, M., \& Epston, D. (1990). \textit{Narrative means to therapeutic ends}. W. W. Norton.

\bibitem{whorf1956} Whorf, B. L. (1956). \textit{Language, thought, and reality: Selected writings of Benjamin Lee Whorf}. MIT Press.

\bibitem{wigner1960} Wigner, E. (1960). The unreasonable effectiveness of mathematics in the natural sciences. \textit{Communications on Pure and Applied Mathematics, 13}(1), 1--14.

\bibitem{wittgenstein2009} Wittgenstein, L. (2009). \textit{Philosophical investigations}. (Orig. 1953). Wiley-Blackwell.

\bibitem{zeeman1976} Zeeman, E. C. (1976). Catastrophe theory. \textit{Scientific American, 234}(4), 65--83.

\end{thebibliography}
